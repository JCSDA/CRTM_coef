\input{preamble.tex}

% Title info
\title{Transmittance Coefficient Generation Package\\ \normalsize{Quickstart Guide}}
\author{Dr. Patrick Stegmann\\ JCSDA}
\date{\today}
\docnumber{(unassigned)}
\docseries{CRTM}


%-------------------------------------------------------------------------------
%                            Ze document begins...
%-------------------------------------------------------------------------------
\begin{document}

\maketitle

%\draftwatermark

\tableofcontents

\newpage

% The front matter
%=================
\thispagestyle{empty}
\vspace*{10cm}
\begin{center}
  {\sffamily\Large\bfseries Change History}
  \begin{table}[htp]
    \centering
    \begin{tabular}{|p{2cm}|p{3cm}|p{8cm}|}
      \hline
      \sffamily\textbf{Date} & \sffamily\textbf{Author} & \sffamily\textbf{Change}\\
      \hline\hline
      2021-12-14 & P. Stegmann& Initial Draft.\\
      \hline
    \end{tabular}
  \end{table}
\end{center}
\clearpage
\pagenumbering{arabic}
\setcounter{page}{1}


% The main matter
%================
%\include{}

\section{Introduction}
This document is a quick start guide for the JCSDA CRTM Transmittance Coefficient Generation package (CGP).
To this end, the document provides a minimal step-by-step guide to create CRTM \verb|SpcCoeff| and \verb|TauCoeff| files for an example MW instrument. 
This is a minimal walkthrough that only demonstrates the mechanics and necessary commands for the end-to-end workflow to obtain instrument coefficients.

\section{CRTM Microwave Coefficient Generation}
The example sensor chosen for this walkthrough is TROPICS Pathfinder, with the CRTM instrument ID \emph{tropics\_sv1\_srf\_v2}.

\subsection{Prerequisites}
The CGP requires a number of depencies that need to be available:
\begin{itemize}
  \item A Fortran compiler, e.g. gfortran or ifort.
  \item netCDF4 and HDF5.
  \item The slurm scheduler for the ODPS regression.
  \item The cmake build system.
  \item git
\end{itemize}

\subsection{Compiling the code}
Before running any of the Fortran applications, they need to be compiled and their dependencies enabled.

\subsubsection{Compiling the CRTM}
In order to make the CRTM itself available to the coefficient generation package, the following steps need to be taken:
\begin{enumerate}
  \item Clone the CRTM repository.
  \item Set up the CRTM environment.
  \item Create a build directory.
  \item Configure the build with \emph{ecbuild}.
  \item Compile the CRTM.
\end{enumerate}
The \emph{bash} terminal commands for the sequence look like this:
\begin{verbatim}
 # Cloning the crtm
 git clone https://github.com/JCSDA-internal/crtm
 cd crtm

 # Setting up the environment
 . ./Set_CRTM_Environnment.sh

 # Creating a build directory
 mkdir build
 cd build

 # Configuring the crtm build
 ecbuild ..

 # Compiling the CRTM
 make -j4
\end{verbatim}
This assumes that you have already cloned and set up the \verb|ecbuild| extension to cmake. For more details please see the documentation of the CRTM and the ECMWF ecbuild user guide.

\subsubsection{Cloning the Coefficient Generation Package}
The CGP can be cloned like this:
\begin{verbatim}
 git clone https://github.com/JCSDA-internal/CRTM_coef
\end{verbatim}

\subsubsection{Creating a new git branch}
Before you create a new instrument coefficient, it is highly encouraged that you create a new branch of the \verb|CRTM_coef| repository for your work, e.g.:
\begin{verbatim}
 git branch feature/tropics
 git checkout feature/tropics
\end{verbatim}

\subsubsection{Configuring the Coefficient Generation Package}
Currently the build system for the CGP is still based on \verb|make|. In order to configure the CGP for the machine you are using, the \verb make.macros file in the CGP root directory needs to be modified. A number of sample macro files are already provided for selected machines, and an example for the \emph{University of Wisconsin SSEC S4} cluster would look like this:
\begin{verbatim}
 # Entering the CGP root directory
 cd CRTM_coef/

 # Linking the machine-specific make macros
 rm make.macros
 ln -s make.macros.s4 make.macros
\end{verbatim}


\subsubsection{Compiling the CGP Fortran Applications}
Finally, all necessary CGP Fortran applications can be compiled. All applications are located in the \verb|src/| folder.
This is a checklist of applications that need to be compiled for a MW sensor:
\begin{todolist}
 \item \verb|oSRF/oSRF_Create_from_ASCII|
 \item \verb|SpcCoeff/Create_SpcCoeff|
 \item \verb|SpcCoeff/SpcCoeff_NC2BIN|
 \item \verb|TauProd/Microwave/MW_TauProfile|
 \item \verb|TauRegress/ODPS/Assemble_ODPS|
 \item \verb|TauRegress/ODPS/GetSenInfo|
 \item \verb|TauRegress/ODPS/ODPS_NC2BIN|
 \item \verb|TauRegress/ODPS/ODPS_Regress|
\end{todolist}
Simply enter each directory and type \verb|make|. Problems can arise at this step if the CRTM or the CGP are not correctly configured, or if one or more of the dependencies are not available.

\subsection{Preparing the SRF}
Before starting the coefficient generation process, the spectral response function of the instrument needs to be brought into the correct input format. 
This is the responsibility of the user, as is making sure that the SRF spectral resolution is high enough to resolve all desired absorption features.
Input data for a new CRTM instrument should be stored in the \verb|inbox/fix/config/| directory of the CGP.
An example ASCII input file for TROPICS looks like this:
\begin{verbatim}
89.484  0.0022256
89.522  0.0026217
89.561  0.0033972
89.599  0.004763
89.638  0.0067661
89.676  0.0095845
89.714  0.01315
89.753  0.016683
89.791  0.018966
89.83   0.020105
89.868  0.02048
89.906  0.019944
89.945  0.0187
89.983  0.017721
90.022  0.019286
90.06   0.024162
90.098  0.028599
90.137  0.031213
90.175  0.031505
90.214  0.029475
...
\end{verbatim}
The first column contains the frequency in units of [GHz] and the second column contains the unitless normalized SRF value.
The example file can be found in \verb|inbox/fix/config/TROPICS_Pathfinder/tropics_sv1_srf_v2/SRFDATA|.
One input file per channel is required.

\subsection{Creating a Case Directory}
The coefficient generation should be carried out in the \verb|workdir| folder.
In the case of a MW sensor, the \verb|workdir/MW| directory is the appropriate location.
In order to create a new case directory, simple type:
\begin{verbatim}
 ./Create_CaseDirectory -i <instrument_name>
\end{verbatim}
where \verb|instrument_name| is the CRTM instrument identifier. This will create the case directory and automatically link in all necessary executables.\\
For the concrete example of TROPICS this would be:
\begin{verbatim}
 ./Create_CaseDirectory -i tropics_sv1_srf_v2
\end{verbatim}
For help on the input options type:
\begin{verbatim}
 ./Create_CaseDirectory -h
\end{verbatim}
Once the case folder is created, you can enter it to proceed.
\begin{verbatim}
 cd tropics_sv1_srf_v2/
\end{verbatim}
Into this folder, link in the sensor information you created from the \verb|inbox/fix/config/| directory.

\subsection{Creating the oSRF file}
The CRTM oSRF file is an intermediate representation of the SRF as a netCDF file. The converter application is \verb|oSRF/oSRF_Create_from_ASCII|.
Currently the executable is called \verb|var.out|. In order to obtain an oSRF file for TROPICS with 1 band, 12 channels and an interpolated SRF resolution of 1 MHz, follow the prompt:
\begin{verbatim}
./var.out 
 Enter the Sensor ID of the current instrument: 
tropics_sv1_srf_v2
 Enter the number of bands for the instrument: 
1
 Enter the number of instrument channels: 
12
 Is SRF interpolation requested (yes/no)?
yes 
 Enter the spectral resolution in GHz: 
0.001
\end{verbatim}
This will create the file \verb|tropics_sv1_srf_v2.osrf.nc|.

\subsection{Creating the SpcCoeff file}
Creating the SpcCoeff file requires the \verb|polangle.txt| and \verb|polarization.txt| files to be linked into the case directory from the \verb|inbox/fix/config/| directory.
They specify the polarization offset angle in degrees per channel and the CRTM polarization scheme per channel as an integer.
In order to create the SpcCoeff spectral coefficient file, run the \verb|Create_SpcCoeff| application in the case directory:
\begin{verbatim}
 ./Create_SpcCoeff 

     **********************************************************
                           Create_SpcCoeff

      Program to create the SpcCoeff files from the sensor 
      oSRF data files.

      $Revision: 46293 $
     **********************************************************


     Enter the oSRF netCDF filename: tropics_sv1_srf_v2.osrf.nc

     Default SpcCoeff version is: 1. Enter value: 2

\end{verbatim}
This will create the file \verb|tropics_sv1_srf_v2.SpcCoeff.nc|.

\subsection{Converting the SpcCoeff file from netCDF to Binary format}
Converting the SpcCoeff file from netCDF to binary format works like this:
\begin{verbatim}
 ./SpcCoeff_NC2BIN 

     **********************************************************
                           SpcCoeff_NC2BIN

      Program to convert a CRTM SpcCoeff data file from netCDF 
      to Binary format.

      $Revision$
     **********************************************************


     Enter the INPUT netCDF SpcCoeff filename : tropics_sv1_srf_v2.SpcCoeff.nc

     Enter the OUTPUT Binary SpcCoeff filename: tropics_sv1_srf_v2.SpcCoeff.bin

     Increment the OUTPUT version number? [y/n]: n
\end{verbatim}
This will create the file \verb|tropics_sv1_srf_v2.SpcCoeff.bin| (either little-endian or big-endian, depending on your system).

\subsection{Rosenkranz MW Transmittance Calculations}
The simplest approach to compute the line-by-line MW layer-to-space transmittances for the ODPS regression is to use the Rosenkranz model implemented in the CRTM.
First, link in the \verb|ECMWF83.AtmProfile.nc| predictor dataset from the \verb|fix/TauProd/| folder.
Second, create a SensorInfo file by modifying an existing template or create a script to write the SensorInfo file for more complex instruments.
A generic SensorInfo generator will be provided in the future.

\begin{verbatim}
 ./MW_TauProfile 

     **********************************************************
                            MW_TauProfile

      Program to compute transmittance profiles for 
      user-defined microwave instruments.

      $Revision: 45710 $
     **********************************************************


     Select atmospheric path
          1) upwelling  
          2) downwelling
     Enter choice: 1

     Enter the SensorInfo filename: SensorInfo_tropics
       Number of sensor entries read: 1

     Enter the netCDF AtmProfile filename: ECMWF83.AtmProfile.nc
\end{verbatim}

Depending on the spectral resolution of the oSRF file, these calculations may take several minutes.

\subsection{ODPS Regression}
The ODPS regression is the last step in the CRTM MW transmittance coefficient generation and produces the TauCoeff file.
This step is performed in a separate folder and an example case directory for the regression is provided in \verb|workdir/MW/GenCoeff/|.
Before starting the regression calculations, the instrument ID needs to be added to the \verb|sensor_list| file:
\begin{verbatim}
 ## This file contains a list of sensors and is one of the input file of
 ## run_tau_coeff.sh
 ## Only those between BEGIN and END are processed.

 BEGIN_LIST
 tropics_sv1_srf_v2
 END_LIST
\end{verbatim}
Afterwards, the \verb|tau_coeff.parameters| input file needs to be modified for your local system:
\begin{verbatim}
:MAX_CPUs:256
:CH_INT:1
:EXE_FILE:/data/users/pstegmann/lib/CRTM/trunk/src/TauRegress/ODPS/ODPS_Regress/Compute_Coeff
:WORK_DIR:CRTM_coef/workdir/MW/GenCoeff
:PROF_SET:ECMWF83
:SPC_COEFF_DIR:CRTM_coef/workdir/MW/tropics_sv1_srf_v2
:TAU_PROFILE_DIR:CRTM_coef/workdir/MW/tropics_sv1_srf_v2
:ATM_PROFLE_FILE:/data/users/pstegmann/lib/CRTM/trunk/fix/TauProd/AtmProfile/ECMWF83.AtmProfile.nc
:GET_SEN_INFO:/data/users/pstegmann/lib/CRTM/trunk/src/TauRegress/ODPS/GetSenInfo/GetSenInfo
:IR_TYPE:3
:COMPONENTS:DEFAULT
#COMPONENTS:ozo
# Parameters below should not be modified in normal operations
:COMPONENT_GROUP1:dry,wlo,wco,ozo,co2,n2o,co,ch4
:COMPONENT_GROUP2:dry,wlo,wco,ozo,co2
:COMPONENT_GROUP3:dry,wet
\end{verbatim}
Make sure that all variables in \verb|tau_coeff.parameters| point to the right directories.
If the \verb|control/| doesn't exist, create it now:
\begin{verbatim}
 mkdir control
\end{verbatim}
Lastly, open the \verb|gen_tau_coeff.sh| script, and check that the log output of the slurm script template points to the \verb|control/| directory:
\begin{verbatim}
echo "#!/bin/bash" >> ${jobScript}
echo "#SBATCH --job-name=ODPS_Run.%j.out" >> ${jobScript}
echo "#SBATCH --partition=serial" >> ${jobScript}
echo "#SBATCH --export=ALL" >> ${jobScript}
#      echo "#SBATCH --share" >> ${jobScript}
#      echo "#SBATCH --account=star" >> ${jobScript}
echo "#SBATCH --time=1:00:00" >> ${jobScript}
echo "#SBATCH --ntasks=1" >> ${jobScript}
echo "#SBATCH --cpus-per-task=1" >> ${jobScript}
echo "#SBATCH --mem-per-cpu=4096" >> ${jobScript}
echo "#SBATCH --output=CRTM_coef/workdir/MW/GenCoeff/control/output.%j.txt" >> ${jobScript}
echo "#SBATCH --error=CRTM_coef/workdir/MW/GenCoeff/control/output.%j.err" >> ${jobScript}
\end{verbatim}

Finally, to perform the regression, open the \verb|run_tau_coeff.sh| script and run step 1 first, followed by step 3:

\begin{verbatim}
#!/bin/sh

#------------------------------------------------------------------------
# Generate transmittance coefficients for each channel and tau component
#------------------------------------------------------------------------

#-----------------------------------
# 1): Generate tau coefficients
#-----------------------------------

#./gen_tau_coeff.sh tau_coeff.parameters

#------------------------------------------------------------------
# 2) Get fitting errors for each component - run after 1)
#------------------------------------------------------------------

#get_stat.sh tau_coeff.parameters

#----------------------------------------------------------------------
# 3) Merge (concatenate) tau coefficient files into one - run after 1)
#----------------------------------------------------------------------

EXE_file=CRTM_coef/src/TauRegress/ODPS/Assemble_ODPS/Cat_ODPS
./cat_taucoef.sh tau_coeff.parameters $EXE_file
\end{verbatim}

If everything worked, the resulting TauCoeff file can be found in the \verb|results/| folder.

\end{document}

