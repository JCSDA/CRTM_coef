\section{oSRF File Generation}

\subsection{Preparations}
The first step in computing CRTM MW transmittance coefficients is to convert the SRF information of the instrument into the CRTM-native \verb|oSRF| data format.
This is done using the \verb|oSRF_Create_from_ASCII| Fortran application:
\begin{verbatim}
cd CRTM_coef/src/oSRF/oSRF_Create_from_ASCII/
make 
\end{verbatim}
The SRF information needs to be placed in the SRFDATA folder. The Fortran code expects one text file per instrument channel.
Each individual text file needs to be formated in this simple manner:
\begin{verbatim}
89.484	0.0022256
89.522	0.0026217
89.561	0.0033972
89.599	0.004763
89.638	0.0067661
89.676	0.0095845
89.714	0.01315
89.753	0.016683
89.791	0.018966
89.83	0.020105
89.868	0.02048
89.906	0.019944
89.945	0.0187
89.983	0.017721
90.022	0.019286
90.06	0.024162
90.098	0.028599
90.137	0.031213
90.175	0.031505
90.214	0.029475
. .
. .
. .
\end{verbatim}
The first column is the SRF frequency in units of Gigahertz [GHz], and the second column is the normalized, dimensionless instrument response $\Phi$.
Thus the first column is the abscissa and the second column is the ordinate of the SRF.
It is the responsibility of the user to prepare the SRF data in the necessary ASCII format and to ensure that the spectral resolution of the SRF is sufficient. 
A convergence study in the spectral domain is highly recommended as part of the coefficient generation process.\\

The name of the ASCII files needs to be specified as the \verb|oSRF_Filename| in the Fortran source file \verb|main.f90| before the application is compiled:
\begin{verbatim}
      ! ============================================================================
      ! **** READ IN THE SRF ASCII DATA ****
      !  
      NULLIFY( head ) ! Initialize list to point to no target

      ios = 0
      Error_Status = 0
      IF( ll < 10 ) THEN
        format_string = "(A42,I1,A4)"
        WRITE(oSRF_Filename,format_string) './SRFDATA/TROPICS_SV1_passband_MIT_LL_Ch0', ll, '.txt'
      ELSE
        format_string = "(A42,I2,A4)"
        WRITE(oSRF_Filename,format_string) './SRFDATA/TROPICS_SV1_passband_MIT_LL_Ch', ll, '.txt'
      END IF
\end{verbatim}

Further metadata for the oSRF format, such as the author, WMO IDs etc. also needs to be specified manually in \verb|main.f90|:
\begin{verbatim}
  ! Create an instance of oSRF_File
  CALL oSRF_File_Create( oSRF_File,n_Channels )
  ! ...Copy over other information
  oSRF_File%Filename         = TRIM(Sensor_Id) // '.osrf.nc'
  oSRF_File%Sensor_ID        = TRIM(Sensor_Id)
  oSRF_File%WMO_Satellite_Id = 1
  oSRF_File%WMO_Sensor_Id    = 1
  oSRF_File%Sensor_Type      = MICROWAVE_SENSOR
  oSRF_File%Title            = 'TROPICS instrument oSRF'
  oSRF_File%History          = 'P. Stegmann, 2020-12-30'
  oSRF_File%Comment          = 'Test implementation'
  oSRF_File%n_Channels       = n_Channels
\end{verbatim}


